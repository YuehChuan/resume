\documentclass{resume}

\usepackage[left=0.75in,top=0.6in,right=0.75in,bottom=0.6in]{geometry}

\name{Yueh Chuan (Johnson), Chang}
\address{ Taichung City, Taiwan 407  }
\address{ schwarmcyc@hotmail.com }

\def\nameskip{\bigskip}
\def\sectionskip{\medskip}

\begin{document}

  \begin{rSection}{Education}
    {\bf National Chiao Tung University, Hsinchu} \hfill {\em June 2014 - July 2018} \\ 
    { M.S. in Electical and Control \& Engineering } \\

    {\bf National Tsing Hua University, Hsinchu} \hfill {\em September 2010 - June 2014} \\ 
    { B.S. in Power Mechanical \& Engineering } \\
  \end{rSection}
  
  \begin{rSection}{Experience}
  
    \begin{rSubsection}{Water Resources Agency, ROC}{October 2018-October 2019}{officer}{Taichung, TW}
    \item Work in goverment department for an one year military service. 
    \end{rSubsection}
	  
    \begin{rSubsection}{Raspberry Pi Taiwan, Inc}{September 2018}{Instructor}{Taipei, TW}
    \item Give a two days hands-on lecture for commercial workshop, Duckietown: Making a learning experience for autonomy project.
    \item 20 students came from different technology companies such as Google, MediaTek, Sunplus Innovation, Ilan university professor.
    \item ref: https://www.raspberrypi.com.tw/21510/duckietown-workshop-01/
    \end{rSubsection}
  
    \begin{rSubsection}{Thesis}{October 2018}{Master Thesis}{Hsinchu, TW}
    \item Topic:  Design and Implementation of a Pose Estimation System Based on Visual Fiducial Features and Multiple Cameras 	    
    \item Published in International Automatic Control Conference 2018, Taoyuan, Taiwan (CACS2018).
    \end{rSubsection}

    \begin{rSubsection}{Aiba-3D Printable Arm Robot}{February 2016 - January 2018}{Artist \& Developer}{Hsinchu, TW}
    \item A proof of concept for building a 3D printable arm robot with stepper motors,without communication and expensive harmonic drive hardware.
    \item Brief talk in ROS conference Taiwan 2018  https://ros-taipei.wixsite.com/2018
    \item Slide: http://bit.ly/hy-yc-habonbon-aiba2018
    \end{rSubsection}
  
  \end{rSection}
  
  \begin{rSection}{Technical Strengths}
    \begin{tabular}{ @{} >{\bfseries}l @{\hspace{6ex}} l }
      Computer Languages & C/C++, Python  \\
	    Protocols \& Libraries & ROS (Robot Operating System), OpenCV, PCL (Point Cloud Library), gtsam\\
      Hardwares & raspberry pi, arduino \\
      Tools & git, CMake
    \end{tabular}
  \end{rSection}

\end{document}
